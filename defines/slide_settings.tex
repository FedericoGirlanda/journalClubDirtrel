% slide_settings

%Praesentationsthemen bestimmen alle Details einer Folie.
%
%\usetheme{
%	AnnArbor | Antibes | Bergen | Berkeley | Berlin | Boadilla | boxes | CambridgeUS | Copenhagen |
%	Darmstadt | default | Dresden | Frankfurt | Goettingen |Hannover | Ilmenau | JuanLesPins | Luebeck |
%	Madrid | Malmoe | Marburg | Montpellier | PaloAlto | Pittsburgh | Rochester | Singapore | Szeged | Warsaw
% }
%Farbthemen bestimmen die Farben einer Folie.
%
%\usecolortheme{
%	albatross | beaver | beetle | crane | default | dolphin | dove | fly | lily | orchid | 
% rose |seagull | seahorse | sidebartab | structure | whale | wolverine
%}	
%Die Themen fuer Schriftzeichen bestimmen die Schriftart.
%
%\usefonttheme{
%	default | professionalfonts | serif | structurebold | structureitalicserif | structuresmallcapsserif
%}
%Innere Themen
%
%Innere Themen spezifizieren die inneren Elemente wie Kopf-, Fusszeile, Sidebar usw. einer Folie.
%
%\useinnertheme{
%	circles | default | inmargin | rectangles | rounded
%}
% aeussere Themen
%
% aeussere Themen spezifizieren die Grenzen einer Folie und sagen ob und wo die inneren Elemente liegen.
%
%\useoutertheme{
%	default | infolines | miniframes | shadow | sidebar | smoothbars | smoothtree | split | tree
%}

\usetheme{Berlin}
\usetheme{Malmoe}
\useoutertheme{infolines}

